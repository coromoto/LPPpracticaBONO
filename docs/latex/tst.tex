\documentclass[spanish,a4paper,11pt]{article}

\usepackage{latexsym,amsfonts,amssymb,amstext,amsthm,float,amsmath}
\usepackage[spanish]{babel}
\usepackage[utf8]{inputenc}
\usepackage[dvips]{epsfig}
%%%%%%%%%%%%%%%%%%%%%%%%%%%%%%%%%%%%%%%%%%%%%%%%%%%%%%%%%%%%%%%%%%%%%%
%123456789012345678901234567890123456789012345678901234567890123456789
%%%%%%%%%%%%%%%%%%%%%%%%%%%%%%%%%%%%%%%%%%%%%%%%%%%%%%%%%%%%%%%%%%%%%%
% Format
%%%%%%%%%%%%%%%%%%%%%%%%%%%%%%%%%%%%%%%%%%%%%%%%%%%%%%%%%%%%%%%%%%%%%%
\topmargin -21 mm
%\headheight 10 mm
%\headsep 10 mm
\textheight 256 mm
\oddsidemargin -5.4 mm
\evensidemargin -5.4 mm
\textwidth 17 cm
%\columnsep 10 mm
%\pagestyle{empty}
%\input{amssym.def}

%%%%%%%%%%%%%%%%%%%%%%%%%%%%%%%%%%%%%%%%%%%%%%%%%%%%%%%%%%%%%%%%%%%%%%
\thispagestyle{empty}
\begin{document}

\begin{tabular}{lccr}
%%%%
  \begin{tabular}{c}
   \epsfig{file=img/ullesc.eps,width=3.5cm} \\
   \tiny{E.S. Ingeniería y Tecnología} \\
   \tiny{Ingeniería Informática} \\
   \tiny{Lenguajes y Sistemas Informáticos}
  \end{tabular}                      &\hspace{1cm} &\hspace{1cm} &
%%%%
  \begin{tabular}{c}
  \large{\textbf{Lenguajes y Paradigmas de Programación}} \\
    Práctica de laboratorio \textbf{*bono*} 
  \end{tabular}        
%%%%
\end{tabular}

\bigskip

\begin{itemize}
\item
El objetivo de este trabajo es elaborar el enunciado de una práctica de laboratorio
en la que se haga patente la diferencia entre las fusiones (\textit{merge}) y 
las reorganizaciones (\textit{rebase}) del sistema de control de versiones \textit{Git}.

\item
Se ha de considerar un proyecto con una rama maestra (\textit{master}), 
una rama \texttt{experimento} y una rama \texttt{sub-experimento}.
Se han de listar los comandos para ponerlo en marcha.
%
En primer lugar se ha de pedir que se realice un ejemplo de fusión (\textit{merge}) 
y que se listen qué confirmaciones (\textit{commits}) hay en cada rama después de la misma.
%
En segundo lugar se ha de pedir que se realice un ejemplo de reorganización (\textit{rebase})
y que se listen qué confirmaciones (\textit{commits}).
%
Además, se tiene que mostrar un ejemplo de lo que ocurre cuando se hace una reorganización
con una rama que se ha hecho pública.


\item
El enunciado de la práctica de laboratorio ha de contener los pasos detallados a seguir, por ejemplo:


\begin{enumerate}
\item
 Iniciar una sesión de trabajo en GNU-Linux.

\item
 Abrir una terminal.

\item
 Situarse en la \textbf{Carpeta de Proyecto} de la asignatura
 Lenguajes y Paradigmas de Programación, esto es, en el directorio \textit{LPP}.

 Uno de los miembros del equipo ha de actuar como \textbf{Coordinador del Equipo de Trabajo}.

 El \textbf{Nombre del Equipo de Trabajo} será: \textsc{LPP}{\_i}
 donde $i$  puede tomar los valores en el rango 1..60, dependiendo del equipo elegido en la consulta.

\item
El coordinador creará la estructura del `directorio de trabajo del equipo'.
%
Transformará el directorio de trabajo local en un `repositorio git'.
%
Creará un `repositorio en \textit{github}' con el nombre del equipo.
%
Dará permisos de escritura en el repositorio de `\textit{github}' a todos los miembros del equipo y a los profesores de prácticas.
%
Creará los ficheros `\verb|.gitignore|' y `\verb|readme.md|'.

\item
Los miembro del equipo `clonarán' el repositorio de \textit{github}.
%
Todos los miembros del equipo, realizarán al menos una confirmación y la incorporarán al repositorio compartido.

... ... ... ... ... ... 

\item
 Cerrar la sesión.

\end{enumerate}


\item
Este trabajo se realizará en equipos de un máximo de tres personas.

\item
El enunciado de la práctica de laboratorio se ha de elaborar utilizando \LaTeX{} y 
ha de ser un proyecto \textit{Git}.

\item
El resultado del trabajo se tiene que presentar en el
\textbf{I Congreso de Estudiantes de Ingeniería Informática}
de la \textsc{ull}, cuyo plazo de presentación de resúmenes es el \textbf{3 de noviembre}.

\end{itemize}

\end{document}
